% This file is part of GUFI, which is part of MarFS, which is released
% under the BSD license.
%
%
% Copyright (c) 2017, Los Alamos National Security (LANS), LLC
% All rights reserved.
%
% Redistribution and use in source and binary forms, with or without modification,
% are permitted provided that the following conditions are met:
%
% 1. Redistributions of source code must retain the above copyright notice, this
% list of conditions and the following disclaimer.
%
% 2. Redistributions in binary form must reproduce the above copyright notice,
% this list of conditions and the following disclaimer in the documentation and/or
% other materials provided with the distribution.
%
% 3. Neither the name of the copyright holder nor the names of its contributors
% may be used to endorse or promote products derived from this software without
% specific prior written permission.
%
% THIS SOFTWARE IS PROVIDED BY THE COPYRIGHT HOLDERS AND CONTRIBUTORS "AS IS" AND
% ANY EXPRESS OR IMPLIED WARRANTIES, INCLUDING, BUT NOT LIMITED TO, THE IMPLIED
% WARRANTIES OF MERCHANTABILITY AND FITNESS FOR A PARTICULAR PURPOSE ARE DISCLAIMED.
% IN NO EVENT SHALL THE COPYRIGHT HOLDER OR CONTRIBUTORS BE LIABLE FOR ANY DIRECT,
% INDIRECT, INCIDENTAL, SPECIAL, EXEMPLARY, OR CONSEQUENTIAL DAMAGES (INCLUDING,
% BUT NOT LIMITED TO, PROCUREMENT OF SUBSTITUTE GOODS OR SERVICES; LOSS OF USE,
% DATA, OR PROFITS; OR BUSINESS INTERRUPTION) HOWEVER CAUSED AND ON ANY THEORY OF
% LIABILITY, WHETHER IN CONTRACT, STRICT LIABILITY, OR TORT (INCLUDING NEGLIGENCE
% OR OTHERWISE) ARISING IN ANY WAY OUT OF THE USE OF THIS SOFTWARE, EVEN IF
% ADVISED OF THE POSSIBILITY OF SUCH DAMAGE.
%
%
% From Los Alamos National Security, LLC:
% LA-CC-15-039
%
% Copyright (c) 2017, Los Alamos National Security, LLC All rights reserved.
% Copyright 2017. Los Alamos National Security, LLC. This software was produced
% under U.S. Government contract DE-AC52-06NA25396 for Los Alamos National
% Laboratory (LANL), which is operated by Los Alamos National Security, LLC for
% the U.S. Department of Energy. The U.S. Government has rights to use,
% reproduce, and distribute this software.  NEITHER THE GOVERNMENT NOR LOS
% ALAMOS NATIONAL SECURITY, LLC MAKES ANY WARRANTY, EXPRESS OR IMPLIED, OR
% ASSUMES ANY LIABILITY FOR THE USE OF THIS SOFTWARE.  If software is
% modified to produce derivative works, such modified software should be
% clearly marked, so as not to confuse it with the version available from
% LANL.
%
% THIS SOFTWARE IS PROVIDED BY LOS ALAMOS NATIONAL SECURITY, LLC AND CONTRIBUTORS
% "AS IS" AND ANY EXPRESS OR IMPLIED WARRANTIES, INCLUDING, BUT NOT LIMITED TO,
% THE IMPLIED WARRANTIES OF MERCHANTABILITY AND FITNESS FOR A PARTICULAR PURPOSE
% ARE DISCLAIMED. IN NO EVENT SHALL LOS ALAMOS NATIONAL SECURITY, LLC OR
% CONTRIBUTORS BE LIABLE FOR ANY DIRECT, INDIRECT, INCIDENTAL, SPECIAL,
% EXEMPLARY, OR CONSEQUENTIAL DAMAGES (INCLUDING, BUT NOT LIMITED TO, PROCUREMENT
% OF SUBSTITUTE GOODS OR SERVICES; LOSS OF USE, DATA, OR PROFITS; OR BUSINESS
% INTERRUPTION) HOWEVER CAUSED AND ON ANY THEORY OF LIABILITY, WHETHER IN
% CONTRACT, STRICT LIABILITY, OR TORT (INCLUDING NEGLIGENCE OR OTHERWISE) ARISING
% IN ANY WAY OUT OF THE USE OF THIS SOFTWARE, EVEN IF ADVISED OF THE POSSIBILITY
% OF SUCH DAMAGE.



\section{Design Considerations}

The following considerations were made to produce a GUFI capability:

\begin{itemize}
\item Why not just flatten the entire metadata entry space and shard
  it and index some fields which enables extremely simple scaling for
  queries?
  \begin{itemize}
  \item Events like rename (mv /top/b2 /top/b1) high in the source
    tree causes potentially billions of records to be
    updated/replaced.
  \item We desire a single parallel index capability used by users and
    admins which requires POSIX sharing security to be enforced which
    is complicated by the inheritance that directory read/execute has
    on the tree below which is one of the powerful concepts of POSIX.
    This security capability is hard to implement in a flat space due
    to the tree inheritance feature.
  \item A single user can see only very little of the overall metadata
    space, simple flat sharding requires looking at a lot of records
    that a \\ tree/graph based approach would eliminate.
  \item Sharding is however important to enable parallelism, just
    simple flat sharding appears to be problematic.
  \end{itemize}
\item Leverage things that work very well, ways to reduce the number
  of records needed to be looked at/updated, etc.
  \begin{itemize}
  \item Just buy product if it exists and if possible without lock in
    etc. that does much of what we want.  We were unsuccessful in
    finding the product to buy.
  \item The POSIX tree walk (directories only (readdir+)) mechanism -
    optimized to the extreme, speed, enables breadth parallel fan out,
    and enforces shared security, enables renames at very low cost
  \item Breadth first search parallelizes extremely well especially
    for threading mechanisms and especially wide namespaces enable
    rapid parallelization which is common in supercomputing sites like
    ours
  \item SQL is extremely powerful and very stable but monolithic
    commercial SQL database systems are often expensive and require
    special knowledge to run. However, user space/embedded SQL
    databases work very well as long as individual database files are
    not enormous (\textless TB) and the application doesn't require a lot of
    joins.  SQLite3 is used heavily in the smart phone business so it
    is becoming quite ubiquitous.  Embedded databases work in POSIX
    file systems appearing as just files so can obey POSIX
    security/access control trivially.  SQL is an amazingly powerful
    way to express queries, more powerful than actually needed for
    this application.  SQLite3 is open source, has enormous support,
    is very fast for this use case, is extensible and allows for
    connecting to other data sources and outputting about any type of
    output.
  \item Flash devices can sustain extremely high IOPs where IOPs are
    in small numbers of kilobytes or larger.
  \item Trees are a natural structure for rolling up representative
    data, so you get indexing function almost for free.
  \item If you consider just the directories (which provide the shape
    of the tree) in most large deployed file/storage systems, there is
    a natural collection of metadata entries (in a single directory)
    of 20-1000s of entries (files/links) giving a nice way to shard
    (by directory) which enables parallelism while still honoring
    POSIX security. In many POSIX file systems, the more files in a
    single directory the worse the performance, but with an embedded
    file system file with few-to-no joins, the more entries in that
    flat file the better as that represents a serial read.
  \item Embedding database files into a POSIX tree allowed for
    replicating the source file/storage system metadata entirely to
    enable an isolated performance domain or placing the index into
    the source storage/file system itself.  This approach also enables
    lots of choices for the underlying file system to store the index
    in and provides trivial ways to backup/copy/replicate the index.
    This approach is trivial to understand and is completely
    transparent.  It also enables the ability for query output to
    appear as a POSIX file system pretty trivially as well.
  \end{itemize}
\end{itemize}
