% This file is part of GUFI, which is part of MarFS, which is released
% under the BSD license.
%
%
% Copyright (c) 2017, Los Alamos National Security (LANS), LLC
% All rights reserved.
%
% Redistribution and use in source and binary forms, with or without modification,
% are permitted provided that the following conditions are met:
%
% 1. Redistributions of source code must retain the above copyright notice, this
% list of conditions and the following disclaimer.
%
% 2. Redistributions in binary form must reproduce the above copyright notice,
% this list of conditions and the following disclaimer in the documentation and/or
% other materials provided with the distribution.
%
% 3. Neither the name of the copyright holder nor the names of its contributors
% may be used to endorse or promote products derived from this software without
% specific prior written permission.
%
% THIS SOFTWARE IS PROVIDED BY THE COPYRIGHT HOLDERS AND CONTRIBUTORS "AS IS" AND
% ANY EXPRESS OR IMPLIED WARRANTIES, INCLUDING, BUT NOT LIMITED TO, THE IMPLIED
% WARRANTIES OF MERCHANTABILITY AND FITNESS FOR A PARTICULAR PURPOSE ARE DISCLAIMED.
% IN NO EVENT SHALL THE COPYRIGHT HOLDER OR CONTRIBUTORS BE LIABLE FOR ANY DIRECT,
% INDIRECT, INCIDENTAL, SPECIAL, EXEMPLARY, OR CONSEQUENTIAL DAMAGES (INCLUDING,
% BUT NOT LIMITED TO, PROCUREMENT OF SUBSTITUTE GOODS OR SERVICES; LOSS OF USE,
% DATA, OR PROFITS; OR BUSINESS INTERRUPTION) HOWEVER CAUSED AND ON ANY THEORY OF
% LIABILITY, WHETHER IN CONTRACT, STRICT LIABILITY, OR TORT (INCLUDING NEGLIGENCE
% OR OTHERWISE) ARISING IN ANY WAY OUT OF THE USE OF THIS SOFTWARE, EVEN IF
% ADVISED OF THE POSSIBILITY OF SUCH DAMAGE.
%
%
% From Los Alamos National Security, LLC:
% LA-CC-15-039
%
% Copyright (c) 2017, Los Alamos National Security, LLC All rights reserved.
% Copyright 2017. Los Alamos National Security, LLC. This software was produced
% under U.S. Government contract DE-AC52-06NA25396 for Los Alamos National
% Laboratory (LANL), which is operated by Los Alamos National Security, LLC for
% the U.S. Department of Energy. The U.S. Government has rights to use,
% reproduce, and distribute this software.  NEITHER THE GOVERNMENT NOR LOS
% ALAMOS NATIONAL SECURITY, LLC MAKES ANY WARRANTY, EXPRESS OR IMPLIED, OR
% ASSUMES ANY LIABILITY FOR THE USE OF THIS SOFTWARE.  If software is
% modified to produce derivative works, such modified software should be
% clearly marked, so as not to confuse it with the version available from
% LANL.
%
% THIS SOFTWARE IS PROVIDED BY LOS ALAMOS NATIONAL SECURITY, LLC AND CONTRIBUTORS
% "AS IS" AND ANY EXPRESS OR IMPLIED WARRANTIES, INCLUDING, BUT NOT LIMITED TO,
% THE IMPLIED WARRANTIES OF MERCHANTABILITY AND FITNESS FOR A PARTICULAR PURPOSE
% ARE DISCLAIMED. IN NO EVENT SHALL LOS ALAMOS NATIONAL SECURITY, LLC OR
% CONTRIBUTORS BE LIABLE FOR ANY DIRECT, INDIRECT, INCIDENTAL, SPECIAL,
% EXEMPLARY, OR CONSEQUENTIAL DAMAGES (INCLUDING, BUT NOT LIMITED TO, PROCUREMENT
% OF SUBSTITUTE GOODS OR SERVICES; LOSS OF USE, DATA, OR PROFITS; OR BUSINESS
% INTERRUPTION) HOWEVER CAUSED AND ON ANY THEORY OF LIABILITY, WHETHER IN
% CONTRACT, STRICT LIABILITY, OR TORT (INCLUDING NEGLIGENCE OR OTHERWISE) ARISING
% IN ANY WAY OUT OF THE USE OF THIS SOFTWARE, EVEN IF ADVISED OF THE POSSIBILITY
% OF SUCH DAMAGE.



\section{Extended Attributes}
Indexing and querying user data is supported through extended
attributes (xattrs).

Reading standard filesystem permissions of files only requires read
(and execute) access to the directory. However, because xattrs are
user defined data, their permissions are checked at the file
level. For more information on xattrs, see \texttt{xattr(7)},
\listxattr, and \getxattr.

\subsection{Indexing}
Directories containing files from multiple users might have xattrs
that are not readable by all who can view the \lstat data of the
directory. As GUFI was originally designed to only use directory-level
permission checks, a number of modifications were made to process
xattrs without violating their permissions.

\subsubsection{Rules for Storage}
Xattrs that are readable by all who have access to the directory are
stored in the \xattrspwd table in the main database. These xattrs are
referred to as ``rolled in''.

The rules that determine whether or not an xattr pair can roll in are
as follows:

\begin{itemize}
\item File is 0+R
\item File is UG+R doesnt matter on other, with file and parent same
  usr and grp and parent has only UG+R with no other read
\item File is U+R doesnt matter on grp and other, with file and parent
  same usr and parent dir has only U+R, no grp and other read
\item Directory has write for every read: drw*rw\_*rw* or
  drw*rw*\_\_\_ or drw*\_\_\_\_\_\_ - if you can write the dir you can
  chmod the files to see the xattrs
\end{itemize}

Xattrs that cannot be read by all who can read the directory are
stored in external per-uid and per-gid databases set with
\texttt{uid:nobdy} and \texttt{nobody:gid} owners respectively. This
makes it so that non-admin users cannot access thse xattrs stored in
the external databases that they do not have permissions to access.

\subsubsection{Schema}
The main database and all external databases contain the following
tables and views:

\begin{itemize}
\item The \xattrspwd table contains all extended attributes of the
  current direcotry that were rolled into the main database file.

\item The \xattrsrollup table contains all extended attributes that
  were rolled into the children directories that were subsequently
  rolled up into the current directory.

\item The \xattrsavail view is the union of all extended attributes
  that were rolled in and rolled up into the main database file.
\end{itemize}

Additionally, each main database file has the following tables and
views in order to keep track of which files were created by GUFI for
the purposes of storing xattrs:

\begin{itemize}
\item The \xattrfilespwd table contains a listing of external database
  filenames that contain extended attributes that were not rolled in.
\item The \xattrfilesrollup table contains a listing of external
  database filenames that contain extended attributes that were not
  rolled in, but were brought in by rolling up.
\item The \xattrfiles view combines the listings of database filenames
  found in \xattrfilespwd and \xattrfilesrollup.
\end{itemize}

\subsubsection{Usage}
Xattrs are not pulled from the filesystem by default. In order to pull
them, pass \texttt{-x} to \gufidirindex or \gufidirtrace.

Note that only xattr pairs whose name starts with the prefix ``user.''
are extracted.

\subsubsection{Rollup}
Rolling up required some additional operations in order to copy xattrs
associated with rolled up files upwards properly:

\begin{itemize}
\item The child's \xattrsavail view (without external database) is
  copied into the parent's \xattrsrollup table.
\item The child's \xattrfiles view (without external databases) is
  copied into the parent's \xattrfilesrollup table.
\item The child's external database files are copied into the
  parent. If the parent already has an external database file with the
  same uid and gid, the contents of the external database are copied
  into the parent's external database's \xattrsrollup table instead.
\end{itemize}

\subsection{Querying}
When querying for xattrs, pass \texttt{-x} to \gufiquery to have it
build the \xattrs view for querying. This view is a SQL union of
rolled in xattrs and any external databases that successfully
attaches. Attaching the database files checks the permissions of the
xattrs. \texttt{UNION} removes duplicate entries that showed up in
multiple external databases, leaving a unique set of xattrs accessible
by the caller.

The \xentries, \xpentries, \xsummary views are convenience views
generated so that users do not have to perform joins themselves. Like
the \xattrs view, these views are also dropped at the end of directory
processing.

Note that entries with multiple extended attributes will return
multiple times in this view.
