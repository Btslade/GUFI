% This file is part of GUFI, which is part of MarFS, which is released
% under the BSD license.
%
%
% Copyright (c) 2017, Los Alamos National Security (LANS), LLC
% All rights reserved.
%
% Redistribution and use in source and binary forms, with or without modification,
% are permitted provided that the following conditions are met:
%
% 1. Redistributions of source code must retain the above copyright notice, this
% list of conditions and the following disclaimer.
%
% 2. Redistributions in binary form must reproduce the above copyright notice,
% this list of conditions and the following disclaimer in the documentation and/or
% other materials provided with the distribution.
%
% 3. Neither the name of the copyright holder nor the names of its contributors
% may be used to endorse or promote products derived from this software without
% specific prior written permission.
%
% THIS SOFTWARE IS PROVIDED BY THE COPYRIGHT HOLDERS AND CONTRIBUTORS "AS IS" AND
% ANY EXPRESS OR IMPLIED WARRANTIES, INCLUDING, BUT NOT LIMITED TO, THE IMPLIED
% WARRANTIES OF MERCHANTABILITY AND FITNESS FOR A PARTICULAR PURPOSE ARE DISCLAIMED.
% IN NO EVENT SHALL THE COPYRIGHT HOLDER OR CONTRIBUTORS BE LIABLE FOR ANY DIRECT,
% INDIRECT, INCIDENTAL, SPECIAL, EXEMPLARY, OR CONSEQUENTIAL DAMAGES (INCLUDING,
% BUT NOT LIMITED TO, PROCUREMENT OF SUBSTITUTE GOODS OR SERVICES; LOSS OF USE,
% DATA, OR PROFITS; OR BUSINESS INTERRUPTION) HOWEVER CAUSED AND ON ANY THEORY OF
% LIABILITY, WHETHER IN CONTRACT, STRICT LIABILITY, OR TORT (INCLUDING NEGLIGENCE
% OR OTHERWISE) ARISING IN ANY WAY OUT OF THE USE OF THIS SOFTWARE, EVEN IF
% ADVISED OF THE POSSIBILITY OF SUCH DAMAGE.
%
%
% From Los Alamos National Security, LLC:
% LA-CC-15-039
%
% Copyright (c) 2017, Los Alamos National Security, LLC All rights reserved.
% Copyright 2017. Los Alamos National Security, LLC. This software was produced
% under U.S. Government contract DE-AC52-06NA25396 for Los Alamos National
% Laboratory (LANL), which is operated by Los Alamos National Security, LLC for
% the U.S. Department of Energy. The U.S. Government has rights to use,
% reproduce, and distribute this software.  NEITHER THE GOVERNMENT NOR LOS
% ALAMOS NATIONAL SECURITY, LLC MAKES ANY WARRANTY, EXPRESS OR IMPLIED, OR
% ASSUMES ANY LIABILITY FOR THE USE OF THIS SOFTWARE.  If software is
% modified to produce derivative works, such modified software should be
% clearly marked, so as not to confuse it with the version available from
% LANL.
%
% THIS SOFTWARE IS PROVIDED BY LOS ALAMOS NATIONAL SECURITY, LLC AND CONTRIBUTORS
% "AS IS" AND ANY EXPRESS OR IMPLIED WARRANTIES, INCLUDING, BUT NOT LIMITED TO,
% THE IMPLIED WARRANTIES OF MERCHANTABILITY AND FITNESS FOR A PARTICULAR PURPOSE
% ARE DISCLAIMED. IN NO EVENT SHALL LOS ALAMOS NATIONAL SECURITY, LLC OR
% CONTRIBUTORS BE LIABLE FOR ANY DIRECT, INDIRECT, INCIDENTAL, SPECIAL,
% EXEMPLARY, OR CONSEQUENTIAL DAMAGES (INCLUDING, BUT NOT LIMITED TO, PROCUREMENT
% OF SUBSTITUTE GOODS OR SERVICES; LOSS OF USE, DATA, OR PROFITS; OR BUSINESS
% INTERRUPTION) HOWEVER CAUSED AND ON ANY THEORY OF LIABILITY, WHETHER IN
% CONTRACT, STRICT LIABILITY, OR TORT (INCLUDING NEGLIGENCE OR OTHERWISE) ARISING
% IN ANY WAY OUT OF THE USE OF THIS SOFTWARE, EVEN IF ADVISED OF THE POSSIBILITY
% OF SUCH DAMAGE.



\subsection{\qptp}
Thread pools are generally written with a single work thread (with a
single lock) from which to threads pull from. This does not work past
a handful of threads. GUFI can run on hundreds of threads, and thus
required a more performant thread pool. The solution to this was
\qptp.

\qptp is named so because a work queue (and lock) are
maintained for each thread in the thread pool.

\subsubsection{Adding Work}
Work is added to threads in a round robin fashion in order to
distribute work evenly and to attempt to prevent, or at least reduce
the amount of, contention experienced by any one work queue.

\subsubsection{Processing Queued Work}
When a thread pops off work for processing, {\bf all} work items
currently enqueued for that thread are removed. This further reduces
the contention the work queues would experience if each work item were
popped off one at a time. All work is processed before the thread
returns to the work queue to find more work.

\subsubsection{Usage}
\begin{enumerate}
\item Create a thread pool: \\ \texttt{struct QPTPool *pool =
  QPTPool\_init(nthreads);}
\item Start the threads: \\ \texttt{QPTPool\_start(pool, args);}
  \\ The \texttt{args} argument will be accessible by all threads that
  are run.
\item Add work: \\ \texttt{QPTPool\_enqueue(pool, 0, function,
  work);} \\ The function passed into \texttt{QPTPool\_enqueue} must
  match the signature found in \texttt{QPTPool.h}. The \texttt{work}
  argument will only be accessible to the thread processing this work.
\item Wait for all work to be completed:
  \\ \texttt{QPTPool\_wait(pool);}
\item Destroy the pool: \\ \texttt{QPTPool\_destroy(pool);}
\end{enumerate}
