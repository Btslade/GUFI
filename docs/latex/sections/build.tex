% This file is part of GUFI, which is part of MarFS, which is released
% under the BSD license.
%
%
% Copyright (c) 2017, Los Alamos National Security (LANS), LLC
% All rights reserved.
%
% Redistribution and use in source and binary forms, with or without modification,
% are permitted provided that the following conditions are met:
%
% 1. Redistributions of source code must retain the above copyright notice, this
% list of conditions and the following disclaimer.
%
% 2. Redistributions in binary form must reproduce the above copyright notice,
% this list of conditions and the following disclaimer in the documentation and/or
% other materials provided with the distribution.
%
% 3. Neither the name of the copyright holder nor the names of its contributors
% may be used to endorse or promote products derived from this software without
% specific prior written permission.
%
% THIS SOFTWARE IS PROVIDED BY THE COPYRIGHT HOLDERS AND CONTRIBUTORS "AS IS" AND
% ANY EXPRESS OR IMPLIED WARRANTIES, INCLUDING, BUT NOT LIMITED TO, THE IMPLIED
% WARRANTIES OF MERCHANTABILITY AND FITNESS FOR A PARTICULAR PURPOSE ARE DISCLAIMED.
% IN NO EVENT SHALL THE COPYRIGHT HOLDER OR CONTRIBUTORS BE LIABLE FOR ANY DIRECT,
% INDIRECT, INCIDENTAL, SPECIAL, EXEMPLARY, OR CONSEQUENTIAL DAMAGES (INCLUDING,
% BUT NOT LIMITED TO, PROCUREMENT OF SUBSTITUTE GOODS OR SERVICES; LOSS OF USE,
% DATA, OR PROFITS; OR BUSINESS INTERRUPTION) HOWEVER CAUSED AND ON ANY THEORY OF
% LIABILITY, WHETHER IN CONTRACT, STRICT LIABILITY, OR TORT (INCLUDING NEGLIGENCE
% OR OTHERWISE) ARISING IN ANY WAY OUT OF THE USE OF THIS SOFTWARE, EVEN IF
% ADVISED OF THE POSSIBILITY OF SUCH DAMAGE.
%
%
% From Los Alamos National Security, LLC:
% LA-CC-15-039
%
% Copyright (c) 2017, Los Alamos National Security, LLC All rights reserved.
% Copyright 2017. Los Alamos National Security, LLC. This software was produced
% under U.S. Government contract DE-AC52-06NA25396 for Los Alamos National
% Laboratory (LANL), which is operated by Los Alamos National Security, LLC for
% the U.S. Department of Energy. The U.S. Government has rights to use,
% reproduce, and distribute this software.  NEITHER THE GOVERNMENT NOR LOS
% ALAMOS NATIONAL SECURITY, LLC MAKES ANY WARRANTY, EXPRESS OR IMPLIED, OR
% ASSUMES ANY LIABILITY FOR THE USE OF THIS SOFTWARE.  If software is
% modified to produce derivative works, such modified software should be
% clearly marked, so as not to confuse it with the version available from
% LANL.
%
% THIS SOFTWARE IS PROVIDED BY LOS ALAMOS NATIONAL SECURITY, LLC AND CONTRIBUTORS
% "AS IS" AND ANY EXPRESS OR IMPLIED WARRANTIES, INCLUDING, BUT NOT LIMITED TO,
% THE IMPLIED WARRANTIES OF MERCHANTABILITY AND FITNESS FOR A PARTICULAR PURPOSE
% ARE DISCLAIMED. IN NO EVENT SHALL LOS ALAMOS NATIONAL SECURITY, LLC OR
% CONTRIBUTORS BE LIABLE FOR ANY DIRECT, INDIRECT, INCIDENTAL, SPECIAL,
% EXEMPLARY, OR CONSEQUENTIAL DAMAGES (INCLUDING, BUT NOT LIMITED TO, PROCUREMENT
% OF SUBSTITUTE GOODS OR SERVICES; LOSS OF USE, DATA, OR PROFITS; OR BUSINESS
% INTERRUPTION) HOWEVER CAUSED AND ON ANY THEORY OF LIABILITY, WHETHER IN
% CONTRACT, STRICT LIABILITY, OR TORT (INCLUDING NEGLIGENCE OR OTHERWISE) ARISING
% IN ANY WAY OUT OF THE USE OF THIS SOFTWARE, EVEN IF ADVISED OF THE POSSIBILITY
% OF SUCH DAMAGE.



\section{Building with CMake}

From the root directory of the GUFI source, run:
\begin{verbatim}
mkdir build
cd build
cmake .. [options]
make
\end{verbatim}

The recommended options for deployment are
\texttt{-DCMAKE\_BUILD\_TYPE=Release -DCLIENT=On}.
\\\\
GUFI is known to build on Ubuntu Xenial, OpenSuse~12.3, CentOS~7,
CentOS~8, and OSX~10.13.

\subsection{Environment Variables}
\begin{table}[h]
\centering
\begin{tabularx}{1.2\textwidth}{| l | X |}
  \hline
  Setting & Description \\
  \hline
  \texttt{CXX=false} & Disable building of C++ code \\
  \hline
\end{tabularx}
\end{table}

\subsection{Flags}

\subsubsection{General}
\begin{table}[h!]
\centering
\begin{tabularx}{1.2\textwidth}{| l | X |}
  \hline
  \texttt{-D<VAR>=<VALUE>} & Description \\
  \hline
  \texttt{CMAKE\_INSTALL\_PREFIX=<PATH>}
  & Install to a custom directory when running \texttt{make install} \\
  \hline
  \texttt{CMAKE\_BUILD\_TYPE=Debug}
  & Build with warnings and debugging symbols turned on \\
  \hline
\end{tabularx}
\end{table}

\subsubsection{Dependencies}
\begin{table}[H]
\centering
\begin{tabularx}{1.2\textwidth}{| l | X |}
  \hline
  \texttt{-D<VAR>=<VALUE>} & Description \\
  \hline
  \texttt{DEP\_DOWNLOAD\_PREFIX=<PATH>}
  & Location of downloaded dependencies. If the expected files are
  found, they will not be downloaded. The default path points to the
  bundled dependencies. \\
  \hline
  \texttt{DEP\_BUILD\_DIR\_PREFIX=<PATH>}
  & Location to build dependencies. Defaults to \\
  & \$\{CMAKE\_BINARY\_DIR\}/builds \\
  \hline
  \texttt{DEP\_INSTALL\_PREFIX=<PATH>}
  & Location to install the dependencies. Defaults to
  \$\{CMAKE\_BINARY\_DIR\}/deps. If the dependencies are not
  installed in \$\{CMAKE\_BINARY\_DIR\}, they will not need to be
  redownloaded, rebuilt, or reinstalled everytime \$\{CMAKE\_BINARY\_DIR\}
  is deleted. \\
  \hline
  \texttt{DEP\_PATCH\_SQLITE3\_OPEN=<On|Off>}
  & Whether or not to patch SQLite3 open \\
  \hline
\end{tabularx}
\end{table}

\subsubsection{Debug}
\texttt{CMAKE\_BUILD\_TYPE} must be set to \texttt{Debug} for these to
have effect.

\begin{table}[H]
\centering
\begin{tabularx}{1.2\textwidth}{| l | X |}
  \hline
  \texttt{-D<VAR>=<VALUE>} & Description \\
  \hline
  \texttt{PRINT\_CUMULATIVE\_TIMES=<On|Off>}
  & Print cumulative statistics at the end of
  some executables. \\
  \hline
  \texttt{PRINT\_PER\_THREAD\_STATS=<On|Off>}
  & Print \texttt{gufi\_query} event timestamps. \\
  \hline
  \texttt{PRINT\_QPTPOOL\_QUEUE\_SIZE=<On|Off>}
  & Print size of work queues every time a work queue receives a new
  item or pops items off. \\
  \hline
  \texttt{PRINT\_SUBDIRECTORY\_COUNTS=<On|Off>}
  & Print the number of subdirectories under each directory. \\
  \hline
  \texttt{GPROF=<On|Off>}
  & Compile with the \texttt{-pg} flag. \\
  \hline
\end{tabularx}
\end{table}

\subsubsection{Client}
\begin{table}[h!]
\centering
\begin{tabularx}{1.2\textwidth}{| l | X |}
  \hline
  \texttt{-D<VAR>=<VALUE>} & Description \\
  \hline
  \texttt{CLIENT=<On|Off>}
  & Whether or not to install paramiko and gufi\_client when \texttt{make install} is called \\
  \hline
\end{tabularx}
\end{table}

\subsubsection{Testing}
Do not turn these off unless you know what you are doing.

\begin{table}[H]
\centering
\begin{tabularx}{1.2\textwidth}{| l | X |}
  \hline
  \texttt{-D<VAR>=<VALUE>} & Description \\
  \hline
  \texttt{RUN\_ADDQUERYFUNCS=<On|Off>}
  & Compile \gufiquery without calls to \texttt{addqueryfuncs}. \\
  \hline
  \texttt{RUN\_OPENDB=<On|Off>}
  & Compile \gufiquery without calls to \texttt{opendb}. \\
  \hline
  \texttt{RUN\_SQL\_EXEC=<On|Off>}
  & Compile \gufiquery without calls to \texttt{sqlite3\_exec}. \\
  \hline
\end{tabularx}
\end{table}

\subsubsection{Benchmarking}
\begin{table}[H]
\centering
\begin{tabularx}{1.2\textwidth}{| l | X |}
  \hline
  \texttt{-D<VAR>=<VALUE>} & Description \\
  \hline
  \texttt{BENCHMARK=<On|Off>}
  & Build so that \gufiquery prints out some benchmarks at the end of
  a run. Make target 'benchmark' is created. This target generates a
  source tree, indexes the tree, and queries the tree. At each step,
  benchmark values will be printed. The tree is deleted after the
  benchmarks complete. \\
  \hline
  \texttt{BENCHMARK\_ROOT=<PATH>}
  & Directory to place benchmark tree and index under \\
  \hline
  \texttt{BENCHMARK\_DIRS=<COUNT>}
  & Default: 4 \\
  \hline
  \texttt{BENCHMARK\_DEPTH=<COUNT>}
  & Default: 10 \\
  \hline
  \texttt{BENCHMARK\_FILES=<COUNT>}
  & Default: 42 \\
  \hline
  \texttt{BENCHMARK\_THREADS=<COUNT>}
  & Default: \# of processors on the machine \\
  \hline
\end{tabularx}
\end{table}
